\documentclass[aspectratio=169]{beamer}
\usetheme{Madrid}
\usecolortheme{dolphin}
\usepackage[utf8]{inputenc}
\usepackage[spanish]{babel}
\usepackage{graphicx}
\usepackage{listings}
\usepackage{fontawesome}
\usepackage{tikz}
\usetikzlibrary{positioning,arrows,shapes}

% Custom colors
\definecolor{cientigo}{RGB}{0, 102, 153}
\setbeamercolor{title}{fg=cientigo}
\setbeamercolor{frametitle}{fg=cientigo}
\setbeamercolor{structure}{fg=cientigo}

% Title information
\title{Infraestructura HPC con SLURM}
\subtitle{Computación de Alto Rendimiento para Investigación}
\author{Pau Santana}
\institute{CientiGO}
\date{\today}

\begin{document}

\begin{frame}
\titlepage
\end{frame}

\begin{frame}
\frametitle{Contenido}
\begin{columns}
\column{0.5\textwidth}
\tableofcontents[sections={1-4}]
\column{0.5\textwidth}
\tableofcontents[sections={5-9}]
\end{columns}

\vspace{0.5cm}
\begin{center}
\textbf{Objetivos de la presentación:}
\begin{itemize}
    \item Comprender la arquitectura del clúster HPC
    \item Conocer las tecnologías clave implementadas
    \item Entender los procedimientos operativos
    \item Explorar casos de uso prácticos
\end{itemize}
\end{center}
\end{frame}

\section{Introducción}

\begin{frame}
\frametitle{Visión General}
\begin{itemize}
    \item \textbf{Propósito:} Infraestructura HPC para investigación científica
    \item \textbf{Componentes clave:}
    \begin{itemize}
        \item SLURM - Gestor de recursos y trabajos
        \item OpenLDAP - Autenticación centralizada
        \item Prometheus/Grafana - Monitorización
        \item Ansible - Automatización y despliegue
    \end{itemize}
    \item \textbf{Beneficios:} Escalabilidad, gestión centralizada, monitorización avanzada
\end{itemize}
\end{frame}

\begin{frame}
\frametitle{Motivación del Proyecto}
\begin{itemize}
    \item \textbf{Necesidades identificadas:}
    \begin{itemize}
        \item Recursos computacionales para investigación avanzada
        \item Gestión eficiente de recursos compartidos
        \item Entorno controlado y seguro
        \item Reducción de costes operativos
    \end{itemize}
    \item \textbf{Solución:} Clúster HPC con gestión centralizada
    \item \textbf{Valor añadido:} Infraestructura como código, reproducible y escalable
\end{itemize}
\end{frame}

\section{Arquitectura}

\begin{frame}
\frametitle{Arquitectura del Sistema}
\begin{columns}
\column{0.5\textwidth}
\textbf{Componentes principales:}
\begin{itemize}
    \item Nodos de control SLURM
    \item Nodos de cómputo
    \item Servidor LDAP
    \item Monitorización
    \item Foreman
\end{itemize}

\column{0.5\textwidth}
\textbf{Topología de red:}
\begin{itemize}
    \item Dominio: \texttt{linkiafp.es}
    \item DNS interno
    \item Servicios distribuidos
    \item Segmentación lógica
\end{itemize}
\end{columns}
\end{frame}

\begin{frame}
\frametitle{Diagrama de Arquitectura}
\centering
\begin{tikzpicture}[node distance=1.5cm, auto]
    % Nodes
    \node[draw, rectangle, rounded corners, fill=blue!20, minimum width=3cm, minimum height=1cm] (slurmctld) {SLURM Controller};
    \node[draw, rectangle, rounded corners, fill=blue!20, minimum width=3cm, minimum height=1cm, below=of slurmctld] (slurmdbd) {SLURM Database};
    \node[draw, rectangle, rounded corners, fill=green!20, minimum width=3cm, minimum height=1cm, left=2cm of slurmctld] (compute) {Compute Nodes};
    \node[draw, rectangle, rounded corners, fill=orange!20, minimum width=3cm, minimum height=1cm, right=2cm of slurmctld] (ldap) {LDAP Server};
    \node[draw, rectangle, rounded corners, fill=purple!20, minimum width=3cm, minimum height=1cm, below=of ldap] (monitoring) {Monitoring};
    
    % Connections
    \draw[->] (slurmctld) -- (slurmdbd);
    \draw[->] (slurmctld) -- (compute);
    \draw[->] (slurmctld) -- (ldap);
    \draw[->] (compute) -- (ldap);
    \draw[->] (monitoring) -- (slurmctld);
    \draw[->] (monitoring) -- (compute);
    \draw[->] (monitoring) -- (ldap);
    \draw[->] (monitoring) -- (slurmdbd);
\end{tikzpicture}
\end{frame}

\begin{frame}[fragile]
\frametitle{Infraestructura Física}
\begin{verbatim}
# Estructura actual
- 1 controlador SLURM (slurm01)
- 1 servidor de base de datos (slurmdb01)
- 2+ nodos de cómputo (nodo01, nodo02, ...)
- 1 nodo de login (login01)
- 1 servidor LDAP (ldap01)
- 1 servidor de servicios (services01)
\end{verbatim}
\end{frame}

\section{Gestión de Usuarios}

\begin{frame}
\frametitle{Sistema LDAP}
\begin{itemize}
    \item \textbf{Estructura:}
    \begin{itemize}
        \item Base DN: \texttt{dc=linkiafp,dc=es}
        \item Usuarios: \texttt{ou=People,dc=linkiafp,dc=es}
        \item Grupos: \texttt{ou=Groups,dc=linkiafp,dc=es}
    \end{itemize}
    \item \textbf{Grupos de investigación:}
    \begin{itemize}
        \item HPC-admins
        \item cfes (Core Facility Earth Sciences)
        \item quantum-physics
        \item particle-physics
        \item astrophysics
        \item condensed-matter
    \end{itemize}
\end{itemize}
\end{frame}

\section{SLURM}

\begin{frame}
\frametitle{Configuración de SLURM}
\begin{columns}
\column{0.5\textwidth}
\textbf{Componentes:}
\begin{itemize}
    \item slurmctld
    \item slurmd
    \item slurmdbd
    \item munge
\end{itemize}

\column{0.5\textwidth}
\textbf{Particiones generales:}
\begin{itemize}
    \item debug
    \item short
    \item medium
    \item long
    \item gpu
\end{itemize}
\end{columns}
\end{frame}

\begin{frame}[fragile]
\frametitle{Particiones por Grupo}
\begin{verbatim}
PARTITION       AVAIL  TIMELIMIT  NODES  STATE NODELIST
cfes-cpu-hswl      up   infinite      1   idle nodo01
qprg-cpu-hswl      up   infinite      1   idle nodo01
pprg-cpu-hswl      up   infinite      1   idle nodo01
arg-cpu-hswl       up   infinite      1   idle nodo01
cmprg-cpu-hswl*    up   infinite      1   idle nodo01
\end{verbatim}

\begin{itemize}
    \item Cada grupo tiene su partición dedicada
    \item Políticas de acceso basadas en pertenencia a grupos
    \item Prioridades configurables por grupo
\end{itemize}
\end{frame}

\begin{frame}[fragile]
\frametitle{Comandos Básicos de SLURM}
\begin{itemize}
    \item \textbf{Envío de trabajos:} 
    \begin{verbatim}
    sbatch script.sh
    srun --partition=debug hostname
    \end{verbatim}
    
    \item \textbf{Monitorización:}
    \begin{verbatim}
    squeue -u $USER
    sinfo -a
    sacct -u $USER --format=JobID,JobName,State
    \end{verbatim}
    
    \item \textbf{Gestión:}
    \begin{verbatim}
    scancel <jobid>
    scontrol show job <jobid>
    \end{verbatim}
\end{itemize}
\end{frame}

\begin{frame}
\frametitle{Políticas de Calidad de Servicio}
\begin{itemize}
    \item \textbf{Límites por usuario:}
    \begin{itemize}
        \item Número máximo de trabajos simultáneos
        \item Tiempo máximo de ejecución por trabajo
        \item Recursos máximos por usuario
    \end{itemize}
    
    \item \textbf{Prioridades:}
    \begin{itemize}
        \item Basadas en grupo de investigación
        \item Ajuste dinámico según uso histórico
        \item Factores de prioridad configurables
    \end{itemize}
    
    \item \textbf{Reservas:} Posibilidad de reservar recursos para eventos especiales
\end{itemize}
\end{frame}

\section{Monitorización}

\begin{frame}
\frametitle{Sistema de Monitorización}
\begin{columns}
\column{0.5\textwidth}
\textbf{Prometheus:}
\begin{itemize}
    \item Recopilación de métricas
    \item Alertas configurables
    \item Datos históricos
    \item Exportadores específicos para SLURM
\end{itemize}

\column{0.5\textwidth}
\textbf{Grafana:}
\begin{itemize}
    \item Dashboards personalizados
    \item Visualización en tiempo real
    \item Informes periódicos
    \item Métricas por grupo/usuario
\end{itemize}
\end{columns}
\end{frame}

\section{Despliegue}

\begin{frame}
\frametitle{Automatización con Ansible}
\begin{itemize}
    \item \textbf{Playbooks principales:}
    \begin{itemize}
        \item \texttt{site.yml} - Playbook principal
        \item \texttt{openldap.yml} - Configuración LDAP
        \item \texttt{slurmctld.yml} - Controlador SLURM
        \item \texttt{compute.yml} - Nodos de cómputo
        \item \texttt{monitoring.yml} - Prometheus/Grafana
    \end{itemize}
    \item \textbf{Ventajas:}
    \begin{itemize}
        \item Despliegue reproducible
        \item Gestión de configuración
        \item Escalabilidad sencilla
        \item Documentación como código
    \end{itemize}
\end{itemize}
\end{frame}

\section{Procedimientos}

\begin{frame}
\frametitle{Procedimientos Operativos}
\begin{columns}
\column{0.5\textwidth}
\textbf{Gestión de usuarios:}
\begin{itemize}
    \item Creación de usuarios LDAP
    \item Asignación a grupos
    \item Gestión de cuotas
    \item Políticas de acceso
\end{itemize}

\column{0.5\textwidth}
\textbf{Mantenimiento:}
\begin{itemize}
    \item Actualizaciones rolling
    \item Backups de configuración
    \item Monitorización proactiva
    \item Gestión de recursos
\end{itemize}
\end{columns}
\end{frame}

\section{Computación Verde}

\begin{frame}
\frametitle{Sostenibilidad Energética}
\begin{itemize}
    \item \textbf{Gestión dinámica de energía:}
    \begin{itemize}
        \item Suspensión de nodos inactivos
        \item Escalado dinámico de frecuencia
        \item Consolidación de trabajos
    \end{itemize}
    \item \textbf{Métricas de sostenibilidad:}
    \begin{itemize}
        \item PUE (Power Usage Effectiveness)
        \item Consumo por trabajo
        \item Huella de carbono
    \end{itemize}
    \item \textbf{Optimizaciones:} Refrigeración eficiente, hardware seleccionado, virtualización
\end{itemize}
\end{frame}

\section{Casos de Uso}

\begin{frame}
\frametitle{Aplicaciones Prácticas}
\begin{columns}
\column{0.5\textwidth}
\textbf{Entorno académico:}
\begin{itemize}
    \item Cursos de computación paralela
    \item Bioinformática
    \item Simulaciones científicas
    \item Laboratorios virtuales
\end{itemize}

\column{0.5\textwidth}
\textbf{Investigación:}
\begin{itemize}
    \item Simulaciones físicas
    \item Análisis de grandes datos
    \item Modelado climático
    \item Cálculos complejos
\end{itemize}
\end{columns}
\end{frame}

\begin{frame}
\frametitle{Conclusiones}
\begin{itemize}
    \item Infraestructura HPC completa y escalable
    \item Gestión centralizada de usuarios y recursos
    \item Monitorización avanzada
    \item Automatización del despliegue
    \item Enfoque en sostenibilidad energética
    \item Adaptable a diversos casos de uso
\end{itemize}

\textbf{Próximos pasos:}
\begin{itemize}
    \item Implementación de alta disponibilidad
    \item Expansión de capacidad de cómputo
    \item Integración con servicios adicionales
\end{itemize}
\end{frame}
\section{Seguridad}

\begin{frame}
\frametitle{Seguridad de la Infraestructura}
\begin{columns}
\column{0.5\textwidth}
\textbf{Autenticación:}
\begin{itemize}
    \item LDAP centralizado
    \item Munge para SLURM
    \item SSH con claves públicas
    \item Políticas de contraseñas
\end{itemize}

\column{0.5\textwidth}
\textbf{Autorización:}
\begin{itemize}
    \item Control de acceso basado en grupos
    \item Permisos granulares
    \item Aislamiento de particiones
    \item Auditoría de acciones
\end{itemize}
\end{columns}
\end{frame}

\begin{frame}
\frametitle{Seguridad en Red}
\begin{itemize}
    \item \textbf{Segmentación de red:}
    \begin{itemize}
        \item Red de administración separada
        \item Red de almacenamiento dedicada
        \item Acceso controlado desde exterior
    \end{itemize}
    \item \textbf{Firewall:} Reglas específicas por servicio
    \item \textbf{Monitorización:} Detección de anomalías
    \item \textbf{Actualizaciones:} Proceso automatizado de parches
\end{itemize}
\end{frame}

\section{Trabajo Futuro}

\begin{frame}
\frametitle{Líneas de Desarrollo Futuro}
\begin{itemize}
    \item \textbf{Alta disponibilidad:}
    \begin{itemize}
        \item Controladores SLURM redundantes
        \item Replicación LDAP
        \item DNS secundario
    \end{itemize}
    
    \item \textbf{Expansión de capacidad:}
    \begin{itemize}
        \item Incorporación de nodos GPU especializados
        \item Ampliación de almacenamiento paralelo
        \item Integración con recursos cloud bajo demanda
    \end{itemize}
    
    \item \textbf{Nuevas funcionalidades:}
    \begin{itemize}
        \item Portal web para usuarios
        \item Integración con sistemas de tickets
        \item Automatización avanzada de workflows científicos
    \end{itemize}
\end{itemize}
\end{frame}

\section{Demostración}

\begin{frame}
\frametitle{Demostración}
\begin{center}
\Large{\textbf{Demostración en vivo}}

\vspace{0.5cm}
\begin{itemize}
    \item Envío de un trabajo a SLURM
    \item Monitorización en Grafana
    \item Gestión de usuarios en LDAP
    \item Despliegue con Ansible
\end{itemize}
\end{center}
\end{frame}
\begin{frame}
\frametitle{¿Preguntas?}
\begin{center}
\Huge{¡Gracias por su atención!}

\vspace{1cm}
\large{Pau Santana}\\
\large{CientiGO}
\end{center}
\end{frame}

\end{document}