\documentclass[12pt,a4paper]{report}

% Paquetes básicos
\usepackage[spanish]{babel}
\usepackage[utf8]{inputenc}
\usepackage[T1]{fontenc}
\usepackage{graphicx}
\usepackage{hyperref}
\usepackage{fancyhdr}
\usepackage{titlesec}
\usepackage{tocloft}
\usepackage{enumitem}
\usepackage{booktabs}
\usepackage{tabularx}
\usepackage{xcolor}
\usepackage{listings}
\usepackage{float}
\usepackage{caption}
\usepackage{subcaption}
\usepackage{appendix}
\usepackage{geometry}
\usepackage{csquotes}
\usepackage{url}
\usepackage{tikz}
\usepackage{pdflscape}
\usepackage{longtable}
\usepackage{tcolorbox}
\usepackage{fontawesome5}
\usepackage{setspace}
\usepackage{array}
\usepackage{multirow}
\usepackage{colortbl}

% Configuración de la página
\geometry{margin=2.5cm}

% Definición de colores de cientiGO
\definecolor{cientigo-blue}{RGB}{0, 82, 155}
\definecolor{cientigo-lightblue}{RGB}{91, 155, 213}
\definecolor{cientigo-green}{RGB}{0, 176, 80}
\definecolor{cientigo-orange}{RGB}{255, 153, 0}
\definecolor{cientigo-gray}{RGB}{128, 128, 128}

% Configuración de encabezado y pie de página
\pagestyle{fancy}
\fancyhf{}
\fancyhead[L]{\textcolor{cientigo-blue}{\leftmark}}
\fancyhead[R]{\textcolor{cientigo-blue}{Automatización de Clústeres HPC con Ansible}}
\fancyfoot[C]{\thepage}
\renewcommand{\headrulewidth}{0.4pt}
\renewcommand{\footrulewidth}{0.4pt}
\renewcommand{\headrule}{\hbox to\headwidth{\color{cientigo-blue}\leaders\hrule height \headrulewidth\hfill}}
\renewcommand{\footrule}{\hbox to\headwidth{\color{cientigo-blue}\leaders\hrule height \footrulewidth\hfill}}

% Configuración de colores para código
\definecolor{codegreen}{rgb}{0,0.6,0}
\definecolor{codegray}{rgb}{0.5,0.5,0.5}
\definecolor{codepurple}{rgb}{0.58,0,0.82}
\definecolor{backcolour}{rgb}{0.95,0.95,0.92}

% Configuración de listings para código
\lstdefinestyle{mystyle}{
    backgroundcolor=\color{backcolour},
    commentstyle=\color{codegreen},
    keywordstyle=\color{cientigo-blue}\bfseries,
    numberstyle=\tiny\color{codegray},
    stringstyle=\color{cientigo-green},
    basicstyle=\ttfamily\small,
    breakatwhitespace=false,
    breaklines=true,
    captionpos=b,
    keepspaces=true,
    numbers=left,
    numbersep=5pt,
    showspaces=false,
    showstringspaces=false,
    showtabs=false,
    tabsize=2,
    frame=single,
    rulecolor=\color{cientigo-blue}
}
\lstset{style=mystyle}

% Definir YAML para listings
\lstdefinelanguage{yaml}{
  keywords={true,false,null,y,n},
  keywordstyle=\color{cientigo-blue}\bfseries,
  basicstyle=\ttfamily\small,
  sensitive=false,
  comment=[l]{\#},
  commentstyle=\color{codegreen},
  stringstyle=\color{cientigo-green},
  morestring=[b]',
  morestring=[b]"
}

% Configuración de títulos
\titleformat{\chapter}[display]
{\normalfont\huge\bfseries\color{cientigo-blue}}
{\chaptertitlename\ \thechapter}{20pt}{\Huge}
\titlespacing*{\chapter}{0pt}{-30pt}{40pt}

\titleformat{\section}
{\normalfont\Large\bfseries\color{cientigo-blue}}
{\thesection}{1em}{}
\titlespacing*{\section}{0pt}{3.5ex plus 1ex minus .2ex}{2.3ex plus .2ex}

\titleformat{\subsection}
{\normalfont\large\bfseries\color{cientigo-lightblue}}
{\thesubsection}{1em}{}

% Configuración de espaciado
\setlength{\parindent}{0pt}
\setlength{\parskip}{8pt}
\onehalfspacing

% Configuración de cajas para destacar información
\newtcolorbox{infocaja}{
  colback=cientigo-lightblue!10,
  colframe=cientigo-blue,
  arc=3mm,
  boxrule=1pt,
  title=\textbf{Información importante},
  fonttitle=\bfseries\color{white},
  coltitle=cientigo-blue,
  attach boxed title to top left={yshift=-2mm, xshift=5mm},
  boxed title style={colback=cientigo-blue}
}

\newtcolorbox{notacaja}{
  colback=cientigo-green!10,
  colframe=cientigo-green,
  arc=3mm,
  boxrule=1pt,
  title=\textbf{Nota},
  fonttitle=\bfseries\color{white},
  coltitle=cientigo-green,
  attach boxed title to top left={yshift=-2mm, xshift=5mm},
  boxed title style={colback=cientigo-green}
}

% Estilo para tablas
\newcommand{\cabeceratabla}{\rowcolor{cientigo-blue}\color{white}\bfseries}

% Información del documento
\title{\Huge{\textcolor{cientigo-blue}{\textbf{Automatización de Clústeres HPC con Ansible}}}\\[0.5cm]
       \Large{\textcolor{cientigo-gray}{Proyecto Técnico - Implementación de Infraestructura como Código}}\\[0.5cm]
       \includegraphics[width=0.4\textwidth]{logo_empresa.png}}
\author{\Large{\textbf{Pau Santana}}}
\date{\today}

\begin{document}

% Portada
\begin{titlepage}
\begin{center}
\vspace*{1cm}
\includegraphics[width=0.6\textwidth]{logo_empresa.png}\\[1.5cm]

\begin{tcolorbox}[
  colback=white,
  colframe=cientigo-blue,
  arc=0mm,
  boxrule=2pt,
  width=\textwidth,
  halign=center,
  valign=center,
  height=3cm
]
\textsc{\LARGE{\textcolor{cientigo-blue}{Ciclo Formativo de Grado Superior}}}\\[0.3cm]
\textsc{\Large{\textcolor{cientigo-gray}{Administración de Sistemas Informáticos en Red}}}
\end{tcolorbox}

\vspace{1cm}

\begin{tcolorbox}[
  colback=cientigo-blue!10,
  colframe=cientigo-blue,
  arc=0mm,
  boxrule=2pt,
  width=\textwidth,
  halign=center,
  valign=center,
  height=5cm
]
{\huge\bfseries\textcolor{cientigo-blue}{Automatización de Clústeres HPC con Ansible}}\\[0.5cm]
{\Large\textcolor{cientigo-gray}{Proyecto Técnico - Implementación de Infraestructura como Código}}
\end{tcolorbox}

\vspace{1cm}

\begin{minipage}{0.45\textwidth}
\begin{tcolorbox}[
  colback=white,
  colframe=cientigo-blue,
  arc=0mm,
  boxrule=1pt,
  width=\textwidth,
  halign=center
]
\large
\textbf{\textcolor{cientigo-blue}{Autor:}}\\
\textcolor{cientigo-gray}{Pau Santana}
\end{tcolorbox}
\end{minipage}
\hfill
\begin{minipage}{0.45\textwidth}
\begin{tcolorbox}[
  colback=white,
  colframe=cientigo-blue,
  arc=0mm,
  boxrule=1pt,
  width=\textwidth,
  halign=center
]
\large
\textbf{\textcolor{cientigo-blue}{Grupo:}}\\
\textcolor{cientigo-gray}{38}
\end{tcolorbox}
\end{minipage}

\vfill

\begin{tcolorbox}[
  colback=white,
  colframe=cientigo-blue,
  arc=0mm,
  boxrule=1pt,
  width=0.5\textwidth,
  halign=center
]
{\large \textcolor{cientigo-gray}{\today}}
\end{tcolorbox}

\end{center}
\end{titlepage}

% Índice
\tableofcontents
\listoffigures
\listoftables
\newpage

% Introducción y resumen en inglés
\chapter*{Introduction and Summary}
\addcontentsline{toc}{chapter}{Introduction and Summary}

\begin{infocaja}
This technical project focuses on the implementation of an automated High-Performance Computing (HPC) cluster using Infrastructure as Code (IaC) principles through Ansible. The primary objective is to create a reproducible, scalable, and energy-efficient computational environment that addresses the critical challenge of scientific reproducibility in computational research.
\end{infocaja}

The project implements a complete HPC environment with Slurm workload manager, LDAP authentication, NFS storage, and integrated monitoring through Prometheus and Grafana. By automating the deployment process, we achieve significant reductions in configuration errors and deployment time while ensuring consistent environments across multiple deployments.

Key objectives include:
\begin{itemize}
    \item \textcolor{cientigo-blue}{\faCheckCircle} Design and implement a fully automated HPC cluster deployment
    \item \textcolor{cientigo-blue}{\faCheckCircle} Create reproducible scientific computing environments
    \item \textcolor{cientigo-blue}{\faCheckCircle} Integrate energy efficiency considerations into the infrastructure
    \item \textcolor{cientigo-blue}{\faCheckCircle} Develop comprehensive monitoring and troubleshooting capabilities
    \item \textcolor{cientigo-blue}{\faCheckCircle} Document the entire process for knowledge transfer and future maintenance
\end{itemize}

The solution addresses a fundamental challenge in computational science by enabling researchers to focus on their scientific work rather than infrastructure management, while ensuring that computational experiments can be reliably reproduced.

\chapter{Introducción}

\section{Resumen del Proyecto}

\begin{tcolorbox}[
  colback=cientigo-blue!5,
  colframe=cientigo-blue,
  arc=2mm,
  boxrule=0.5pt
]
Este proyecto técnico se centra en la implementación de un clúster de Computación de Alto Rendimiento (HPC) automatizado utilizando principios de Infraestructura como Código (IaC) a través de Ansible. El objetivo principal es crear un entorno computacional reproducible, escalable y energéticamente eficiente que aborde el desafío crítico de la reproducibilidad científica en la investigación computacional.
\end{tcolorbox}

El proyecto implementa un entorno HPC completo con el gestor de cargas de trabajo Slurm, autenticación LDAP, almacenamiento NFS y monitorización integrada a través de Prometheus y Grafana. Al automatizar el proceso de despliegue, logramos reducciones significativas en errores de configuración y tiempo de implementación, asegurando entornos consistentes en múltiples despliegues.

\section{Objetivos del Proyecto}

Los objetivos principales de este proyecto son:

\begin{itemize}
    \item[\textcolor{cientigo-blue}{\faTarget}] Diseñar e implementar un despliegue de clúster HPC completamente automatizado
    \item[\textcolor{cientigo-blue}{\faTarget}] Crear entornos de computación científica reproducibles
    \item[\textcolor{cientigo-blue}{\faTarget}] Integrar consideraciones de eficiencia energética en la infraestructura
    \item[\textcolor{cientigo-blue}{\faTarget}] Desarrollar capacidades completas de monitorización y resolución de problemas
    \item[\textcolor{cientigo-blue}{\faTarget}] Documentar todo el proceso para la transferencia de conocimientos y el mantenimiento futuro
\end{itemize}

\begin{notacaja}
La solución aborda un desafío fundamental en la ciencia computacional al permitir que los investigadores se centren en su trabajo científico en lugar de la gestión de infraestructura, garantizando que los experimentos computacionales puedan reproducirse de manera confiable.
\end{notacaja}

\section{Justificación del Proyecto}

En el entorno actual de investigación científica y computación de alto rendimiento, la reproducibilidad de los resultados computacionales se ha convertido en un desafío crítico. Muchos estudios científicos no pueden ser reproducidos debido a la falta de documentación adecuada sobre el entorno computacional utilizado. Este proyecto aborda directamente este problema mediante la automatización completa de la infraestructura, permitiendo que los entornos computacionales sean versionados, documentados y reproducidos con precisión.

Además, la gestión manual de clústeres HPC es propensa a errores, consume mucho tiempo y requiere conocimientos especializados. La automatización no solo reduce estos problemas, sino que también mejora la eficiencia operativa, reduce los costos y permite una mejor utilización de los recursos computacionales.

\chapter{Planificación Temporal y Asignación de Recursos}

\section{Cronograma del Proyecto}

El proyecto se ha dividido en varias fases con una duración total estimada de 12 semanas. A continuación se presenta el cronograma detallado:

\begin{table}[H]
\centering
\renewcommand{\arraystretch}{1.3}
\begin{tabular}{|>{\columncolor{cientigo-blue!10}}l|l|c|l|}
\hline
\cabeceratabla \textbf{Fase} & \textbf{Tareas} & \textbf{Duración} & \textbf{Responsable} \\
\hline
\multirow{3}{*}{Análisis y Diseño} & Análisis de requisitos & \multirow{3}{*}{1 semana} & \multirow{3}{*}{Pau Santana} \\
 & Diseño de arquitectura & & \\
 & Selección de tecnologías & & \\
\hline
\rowcolor{cientigo-blue!5}
\multirow{3}{*}{Implementación Base} & Configuración de Foreman & \multirow{3}{*}{1 semana} & \multirow{3}{*}{Pau Santana} \\
\rowcolor{cientigo-blue!5}
 & Desarrollo de playbooks Ansible & & \\
\rowcolor{cientigo-blue!5}
 & Implementación de LDAP & & \\
\hline
\multirow{3}{*}{Implementación Slurm} & Configuración de Slurm Controller & \multirow{3}{*}{2 semanas} & \multirow{3}{*}{Pau Santana} \\
 & Configuración de Slurm DB & & \\
 & Configuración de nodos de cómputo & & \\
\hline
\rowcolor{cientigo-blue!5}
\multirow{3}{*}{Almacenamiento y Software} & Configuración de NFS & \multirow{3}{*}{2 semanas} & \multirow{3}{*}{Pau Santana} \\
\rowcolor{cientigo-blue!5}
 & Implementación de Spack & & \\
\rowcolor{cientigo-blue!5}
 & Optimización de software científico & & \\
\hline
\multirow{3}{*}{Monitorización} & Implementación de Prometheus & \multirow{3}{*}{1 semana} & \multirow{3}{*}{Pau Santana} \\
 & Configuración de Grafana & & \\
 & Desarrollo de dashboards & & \\
\hline
\rowcolor{cientigo-blue!5}
\multirow{3}{*}{Pruebas y Optimización} & Pruebas de rendimiento & \multirow{3}{*}{1 semana} & \multirow{3}{*}{Pau Santana} \\
\rowcolor{cientigo-blue!5}
 & Optimización energética & & \\
\rowcolor{cientigo-blue!5}
 & Validación de reproducibilidad & & \\
\hline
\multirow{3}{*}{Documentación} & Elaboración de manuales & \multirow{3}{*}{1 semana} & \multirow{3}{*}{Pau Santana} \\
 & Documentación técnica & & \\
 & Memoria del proyecto & & \\
\hline
\end{tabular}
\caption{Cronograma del proyecto}
\label{tab:cronograma}
\end{table}

\section{Recursos Necesarios}

\subsection{Recursos Humanos}

Para este proyecto se requiere un administrador de sistemas con conocimientos en:
\begin{itemize}
    \item[\textcolor{cientigo-blue}{\faUser}] Administración de sistemas Linux
    \item[\textcolor{cientigo-blue}{\faUser}] Automatización con Ansible
    \item[\textcolor{cientigo-blue}{\faUser}] Gestores de carga de trabajo HPC (Slurm)
    \item[\textcolor{cientigo-blue}{\faUser}] Servicios de directorio (LDAP)
    \item[\textcolor{cientigo-blue}{\faUser}] Sistemas de monitorización (Prometheus/Grafana)
    \item[\textcolor{cientigo-blue}{\faUser}] Virtualización y contenedores
\end{itemize}

\subsection{Recursos Hardware}

\begin{table}[H]
\centering
\renewcommand{\arraystretch}{1.3}
\begin{tabular}{|>{\columncolor{cientigo-blue!10}}l|c|p{3.5cm}|p{5.5cm}|}
\hline
\cabeceratabla \textbf{Componente} & \textbf{Cantidad} & \textbf{Especificaciones} & \textbf{Propósito} \\
\hline
Servidor de gestión & 1 & 8 cores, 32GB RAM, 500GB SSD & Foreman, Ansible, LDAP \\
\hline
\rowcolor{cientigo-blue!5}
Nodo controlador & 1 & 16 cores, 64GB RAM, 1TB SSD & Slurm Controller, DB \\
\hline
Nodos de cómputo & 4 & 32 cores, 128GB RAM, 2TB SSD & Procesamiento HPC \\
\hline
\rowcolor{cientigo-blue!5}
Servidor de almacenamiento & 1 & 8 cores, 32GB RAM, 10TB RAID & NFS, Backups \\
\hline
Switches de red & 2 & 24 puertos, 10GbE & Interconexión \\
\hline
\end{tabular}
\caption{Recursos hardware necesarios}
\label{tab:hardware}
\end{table}

\subsection{Recursos Software}

\begin{table}[H]
\centering
\renewcommand{\arraystretch}{1.3}
\begin{tabular}{|>{\columncolor{cientigo-blue!10}}l|l|l|}
\hline
\cabeceratabla \textbf{Software} & \textbf{Versión} & \textbf{Propósito} \\
\hline
CentOS/Rocky Linux & 8.x & Sistema operativo base \\
\hline
\rowcolor{cientigo-blue!5}
Ansible & 2.9+ & Automatización de infraestructura \\
\hline
Foreman & 3.x & Aprovisionamiento de servidores \\
\hline
\rowcolor{cientigo-blue!5}
Slurm & 21.08+ & Gestor de cargas de trabajo HPC \\
\hline
OpenLDAP & 2.4+ & Autenticación centralizada \\
\hline
\rowcolor{cientigo-blue!5}
Prometheus & 2.x & Monitorización de sistemas \\
\hline
Grafana & 8.x & Visualización de métricas \\
\hline
\rowcolor{cientigo-blue!5}
Spack & 0.17+ & Gestor de paquetes científicos \\
\hline
\end{tabular}
\caption{Recursos software necesarios}
\label{tab:software}
\end{table}

\chapter{Presentación Teórica de Tecnologías}

\section{Infraestructura como Código (IaC)}

\begin{tcolorbox}[
  colback=cientigo-blue!5,
  colframe=cientigo-blue,
  arc=2mm,
  boxrule=0.5pt,
  title=\textbf{Infraestructura como Código},
  fonttitle=\bfseries\color{white},
  coltitle=cientigo-blue,
  attach boxed title to top center={yshift=-2mm},
  boxed title style={colback=cientigo-blue}
]
La Infraestructura como Código (IaC) es un enfoque para la gestión y aprovisionamiento de infraestructura a través de archivos de definición en lugar de procesos manuales. Este paradigma permite tratar la infraestructura de la misma manera que los desarrolladores tratan el código de aplicación, aplicando las mismas prácticas de desarrollo de software como control de versiones, integración continua, revisión de código y pruebas automatizadas.
\end{tcolorbox}

\subsection{Beneficios de IaC}

\begin{itemize}
    \item[\textcolor{cientigo-green}{\faCheck}] \textbf{Reproducibilidad:} La infraestructura puede ser recreada de manera idéntica en diferentes entornos.
    \item[\textcolor{cientigo-green}{\faCheck}] \textbf{Escalabilidad:} Facilita la creación y gestión de múltiples entornos similares.
    \item[\textcolor{cientigo-green}{\faCheck}] \textbf{Consistencia:} Reduce las variaciones entre entornos de desarrollo, prueba y producción.
    \item[\textcolor{cientigo-green}{\faCheck}] \textbf{Velocidad:} Automatiza procesos que de otra manera serían manuales y propensos a errores.
    \item[\textcolor{cientigo-green}{\faCheck}] \textbf{Documentación:} El código sirve como documentación viva de la infraestructura.
    \item[\textcolor{cientigo-green}{\faCheck}] \textbf{Control de versiones:} Permite rastrear cambios y revertir a estados anteriores si es necesario.
\end{itemize}

\subsection{Enfoques de IaC}

Existen dos enfoques principales para IaC:

\begin{table}[H]
\centering
\renewcommand{\arraystretch}{1.3}
\begin{tabular}{|>{\columncolor{cientigo-blue!10}}l|p{10cm}|}
\hline
\cabeceratabla \textbf{Enfoque} & \textbf{Descripción} \\
\hline
\textbf{Declarativo (funcional)} & Se especifica el estado deseado de la infraestructura, y la herramienta determina cómo alcanzarlo. Ejemplos: Terraform, CloudFormation, ARM Templates. \\
\hline
\rowcolor{cientigo-blue!5}
\textbf{Imperativo (procedimental)} & Se especifican los comandos exactos necesarios para alcanzar el estado deseado. Ejemplos: Scripts de shell, algunas implementaciones de Ansible. \\
\hline
\end{tabular}
\end{table}

\begin{notacaja}
Ansible, la herramienta principal utilizada en este proyecto, puede funcionar en ambos modos, aunque tiende a favorecer el enfoque declarativo.
\end{notacaja}

\section{Ansible}

Ansible es una herramienta de automatización de TI de código abierto que permite la configuración de sistemas, el despliegue de software y la orquestación de tareas más avanzadas como despliegues continuos o actualizaciones sin tiempo de inactividad.

\subsection{Arquitectura de Ansible}

\begin{figure}[H]
\centering
% Espacio para captura de pantalla de la arquitectura de Ansible
\caption{Arquitectura de Ansible}
\label{fig:ansible_architecture}
\end{figure}

Ansible utiliza una arquitectura sin agentes, lo que significa que no requiere la instalación de software adicional en los nodos gestionados. En su lugar, utiliza SSH para conectarse a los nodos y ejecutar los módulos necesarios. Esta arquitectura consta de:

\begin{itemize}
    \item[\textcolor{cientigo-blue}{\faCube}] \textbf{Nodo de control:} El sistema donde se instala y ejecuta Ansible.
    \item[\textcolor{cientigo-blue}{\faCube}] \textbf{Inventario:} Define los hosts y grupos de hosts que Ansible puede gestionar.
    \item[\textcolor{cientigo-blue}{\faCube}] \textbf{Playbooks:} Archivos YAML que definen las tareas a ejecutar en los hosts.
    \item[\textcolor{cientigo-blue}{\faCube}] \textbf{Roles:} Organizan los playbooks y otros archivos relacionados para facilitar la reutilización.
    \item[\textcolor{cientigo-blue}{\faCube}] \textbf{Módulos:} Unidades de código que Ansible ejecuta en los hosts remotos.
    \item[\textcolor{cientigo-blue}{\faCube}] \textbf{Plugins:} Extienden la funcionalidad de Ansible.
\end{itemize}

\subsection{Conceptos Clave de Ansible}

\begin{tcolorbox}[
  enhanced,
  colback=white,
  colframe=cientigo-blue,
  arc=0mm,
  boxrule=1pt,
  title=Conceptos Clave de Ansible,
  attach boxed title to top center={yshift=-\tcboxedtitleheight/2},
  boxed title style={
    colback=cientigo-blue,
    colframe=cientigo-blue,
    fontupper=\bfseries\color{white},
  },
  coltitle=white,
  fonttitle=\bfseries
]
\begin{itemize}
    \item \textbf{Idempotencia:} Las operaciones pueden aplicarse varias veces sin cambiar el resultado más allá de la aplicación inicial.
    \item \textbf{Playbooks:} Definen una serie de tareas a ejecutar en hosts específicos.
    \item \textbf{Roles:} Permiten organizar playbooks y otros archivos para facilitar la reutilización.
    \item \textbf{Variables:} Permiten personalizar playbooks para diferentes entornos o hosts.
    \item \textbf{Templates:} Utilizan Jinja2 para generar archivos de configuración dinámicos.
    \item \textbf{Handlers:} Tareas que solo se ejecutan cuando son notificadas por otras tareas.
\end{itemize}
\end{tcolorbox}

\subsection{Ejemplo de Playbook}

\begin{lstlisting}[language=yaml]
---
- name: Configurar servidor LDAP
  hosts: ldap_servers
  become: yes
  vars:
    ldap_domain: "example.com"
    ldap_organization: "Example Inc"
    ldap_admin_password: "{{ vault_ldap_admin_password }}"
  
  tasks:
    - name: Instalar paquetes OpenLDAP
      package:
        name:
          - openldap
          - openldap-servers
          - openldap-clients
        state: present
    
    - name: Iniciar y habilitar servicio slapd
      service:
        name: slapd
        state: started
        enabled: yes
    
    - name: Configurar dominio LDAP
      template:
        src: templates/ldap.conf.j2
        dest: /etc/openldap/ldap.conf
      notify: Reiniciar slapd
  
  handlers:
    - name: Reiniciar slapd
      service:
        name: slapd
        state: restarted
\end{lstlisting}

\section{Computación de Alto Rendimiento (HPC)}

\begin{tcolorbox}[
  colback=cientigo-blue!5,
  colframe=cientigo-blue,
  arc=2mm,
  boxrule=0.5pt
]
La Computación de Alto Rendimiento (HPC, por sus siglas en inglés) se refiere al uso de supercomputadoras y clústeres de computación para resolver problemas complejos que requieren gran capacidad de procesamiento o memoria. HPC es fundamental en campos como la simulación física, el modelado climático, la genómica, el aprendizaje automático y muchas otras áreas de investigación científica.
\end{tcolorbox}

\subsection{Componentes de un Clúster HPC}

Un clúster HPC típico consta de los siguientes componentes:

\begin{itemize}
    \item[\textcolor{cientigo-orange}{\faServer}] \textbf{Nodos de cómputo:} Servidores dedicados al procesamiento de cálculos.
    \item[\textcolor{cientigo-orange}{\faServer}] \textbf{Nodo controlador:} Gestiona el clúster y distribuye las tareas.
    \item[\textcolor{cientigo-orange}{\faNetworkWired}] \textbf{Red de interconexión:} Proporciona comunicación de alta velocidad entre nodos.
    \item[\textcolor{cientigo-orange}{\faDatabase}] \textbf{Sistema de almacenamiento:} Proporciona acceso a datos para los cálculos.
    \item[\textcolor{cientigo-orange}{\faListAlt}] \textbf{Sistema de gestión de colas:} Software que programa y asigna recursos a los trabajos.
    \item[\textcolor{cientigo-orange}{\faCode}] \textbf{Software científico:} Aplicaciones específicas para diferentes dominios científicos.
\end{itemize}

\subsection{Arquitectura de un Clúster HPC}

\begin{figure}[H]
\centering
\begin{tikzpicture}[node distance=1.2cm, auto, thick, scale=0.9, transform shape]
    % Definir estilos
    \tikzstyle{block} = [rectangle, draw=cientigo-blue, fill=cientigo-blue!10, text width=2.8cm, text centered, rounded corners, minimum height=2em]
    \tikzstyle{line} = [draw, -latex']
    
    % Nodos
    \node [block] (controller) {Nodo Controlador};
    \node [block, below left=of controller] (compute1) {Nodo de Cómputo 1};
    \node [block, below=of controller] (compute2) {Nodo de Cómputo 2};
    \node [block, below right=of controller] (compute3) {Nodo de Cómputo 3};
    \node [block, right=of controller] (storage) {Almacenamiento};
    \node [block, left=of controller] (login) {Nodo de Login};
    
    % Conexiones
    \path [line] (login) -- (controller);
    \path [line] (controller) -- (compute1);
    \path [line] (controller) -- (compute2);
    \path [line] (controller) -- (compute3);
    \path [line] (controller) -- (storage);
    \path [line] (compute1) -- (storage);
    \path [line] (compute2) -- (storage);
    \path [line] (compute3) -- (storage);
\end{tikzpicture}
\caption{Arquitectura típica de un clúster HPC}
\label{fig:hpc_architecture}
\end{figure}

\section{Slurm Workload Manager}

\begin{tcolorbox}[
  colback=cientigo-blue!5,
  colframe=cientigo-blue,
  arc=2mm,
  boxrule=0.5pt,
  title=\textbf{Slurm Workload Manager},
  fonttitle=\bfseries\color{white},
  coltitle=cientigo-blue,
  attach boxed title to top center={yshift=-2mm},
  boxed title style={colback=cientigo-blue}
]
Slurm (Simple Linux Utility for Resource Management) es un sistema de gestión de cargas de trabajo de código abierto diseñado para clústeres Linux de todos los tamaños. Es utilizado en muchos de los supercomputadores más grandes del mundo y proporciona tres funciones clave:
\end{tcolorbox}

\begin{itemize}
    \item[\textcolor{cientigo-green}{\faCheck}] Asignar acceso exclusivo y/o no exclusivo a recursos a los usuarios durante un tiempo determinado.
    \item[\textcolor{cientigo-green}{\faCheck}] Proporcionar un marco para iniciar, ejecutar y monitorizar trabajos.
    \item[\textcolor{cientigo-green}{\faCheck}] Arbitrar conflictos por recursos mediante la gestión de una cola de trabajos pendientes.
\end{itemize}

\subsection{Componentes de Slurm}

\begin{itemize}
    \item[\textcolor{cientigo-blue}{\faCogs}] \textbf{slurmctld:} El controlador central que gestiona el estado del clúster y asigna recursos.
    \item[\textcolor{cientigo-blue}{\faCog}] \textbf{slurmd:} Demonio que se ejecuta en cada nodo de cómputo para ejecutar trabajos.
    \item[\textcolor{cientigo-blue}{\faDatabase}] \textbf{slurmdbd:} Demonio de base de datos que registra información contable.
    \item[\textcolor{cientigo-blue}{\faTerminal}] \textbf{cliente:} Comandos como \texttt{srun}, \texttt{sbatch}, \texttt{squeue} que permiten a los usuarios interactuar con Slurm.
\end{itemize}

\subsection{Arquitectura de Slurm}

\begin{figure}[H]
\centering
\begin{tikzpicture}[node distance=1.2cm, auto, thick, scale=0.9, transform shape]
    % Definir estilos
    \tikzstyle{block} = [rectangle, draw=cientigo-blue, fill=cientigo-blue!10, text width=2.8cm, text centered, rounded corners, minimum height=2em]
    \tikzstyle{line} = [draw, -latex']
    
    % Nodos
    \node [block] (slurmctld) {slurmctld};
    \node [block, below left=of slurmctld] (slurmd1) {slurmd (nodo 1)};
    \node [block, below=of slurmctld] (slurmd2) {slurmd (nodo 2)};
    \node [block, below right=of slurmctld] (slurmd3) {slurmd (nodo 3)};
    \node [block, right=of slurmctld] (slurmdbd) {slurmdbd};
    \node [block, right=of slurmdbd] (database) {Base de datos};
    \node [block, left=of slurmctld] (client) {Cliente};
    
    % Conexiones
    \path [line] (client) -- (slurmctld);
    \path [line] (slurmctld) -- (slurmd1);
    \path [line] (slurmctld) -- (slurmd2);
    \path [line] (slurmctld) -- (slurmd3);
    \path [line] (slurmctld) -- (slurmdbd);
    \path [line] (slurmdbd) -- (database);
\end{tikzpicture}
\caption{Arquitectura de Slurm Workload Manager}
\label{fig:slurm_architecture}
\end{figure}

\subsection{Ejemplo de Script de Trabajo Slurm}

\begin{lstlisting}[language=bash]
#!/bin/bash
#SBATCH --job-name=ejemplo_hpc
#SBATCH --output=resultado_%j.out
#SBATCH --error=error_%j.err
#SBATCH --ntasks=16
#SBATCH --nodes=2
#SBATCH --time=01:00:00
#SBATCH --mem=32G

# Cargar módulos necesarios
module load openmpi/4.1.1
module load python/3.9

# Ejecutar aplicación paralela
mpirun -np $SLURM_NTASKS python3 simulacion.py
\end{lstlisting}

\section{LDAP (Lightweight Directory Access Protocol)}

\begin{tcolorbox}[
  colback=cientigo-blue!5,
  colframe=cientigo-blue,
  arc=2mm,
  boxrule=0.5pt,
  title=\textbf{LDAP},
  fonttitle=\bfseries\color{white},
  coltitle=cientigo-blue,
  attach boxed title to top center={yshift=-2mm},
  boxed title style={colback=cientigo-blue}
]
LDAP es un protocolo de aplicación abierto, independiente del proveedor, para acceder y mantener servicios de información de directorio distribuidos sobre una red IP. Se utiliza principalmente como un servicio de directorio centralizado para una organización, almacenando información sobre usuarios, grupos y otros objetos.
\end{tcolorbox}

\subsection{Estructura de LDAP}

LDAP organiza la información en una estructura jerárquica llamada Directorio de Información (DIT). Cada entrada en el directorio tiene un Nombre Distinguido (DN) único y consiste en una colección de atributos.

\begin{figure}[H]
\centering
\begin{tikzpicture}[
    level 1/.style={sibling distance=40mm},
    level 2/.style={sibling distance=20mm},
    level 3/.style={sibling distance=20mm},
    every node/.style={draw=cientigo-blue, fill=cientigo-blue!10, rounded corners, text centered}
]
    \node {dc=example,dc=com}
        child {node {ou=People}
            child {node {uid=user1}}
            child {node {uid=user2}}
            child {node {uid=user3}}
        }
        child {node {ou=Groups}
            child {node {cn=admins}}
            child {node {cn=users}}
            child {node {cn=hpc}}
        };
\end{tikzpicture}
\caption{Estructura jerárquica de LDAP}
\label{fig:ldap_structure}
\end{figure}

\subsection{Integración de LDAP con HPC}

\begin{tcolorbox}[
  enhanced,
  colback=white,
  colframe=cientigo-green,
  arc=0mm,
  boxrule=1pt,
  title=Integración LDAP-HPC,
  attach boxed title to top center={yshift=-\tcboxedtitleheight/2},
  boxed title style={
    colback=cientigo-green,
    colframe=cientigo-green,
    fontupper=\bfseries\color{white},
  },
  coltitle=white,
  fonttitle=\bfseries
]
En un entorno HPC, LDAP proporciona:
\begin{itemize}
    \item \textbf{Autenticación centralizada} para todos los nodos del clúster
    \item \textbf{Gestión unificada} de usuarios y grupos
    \item \textbf{Control de acceso} a recursos del clúster
    \item \textbf{Sincronización} de UIDs y GIDs entre nodos
\end{itemize}
\end{tcolorbox}

\section{Sistemas de Monitorización: Prometheus y Grafana}

\subsection{Prometheus}

\begin{tcolorbox}[
  colback=cientigo-orange!10,
  colframe=cientigo-orange,
  arc=2mm,
  boxrule=0.5pt,
  title=\textbf{Prometheus},
  fonttitle=\bfseries\color{white},
  coltitle=cientigo-orange,
  attach boxed title to top center={yshift=-2mm},
  boxed title style={colback=cientigo-orange}
]
Prometheus es un sistema de monitorización y alerta de código abierto, originalmente desarrollado por SoundCloud. Se centra en la fiabilidad y está diseñado para recopilar métricas en tiempo real de sistemas objetivo a través de un modelo de extracción.
\end{tcolorbox}

Características principales:
\begin{itemize}
    \item[\textcolor{cientigo-blue}{\faChartLine}] Modelo de datos multidimensional con series temporales identificadas por nombre de métrica y pares clave-valor
    \item[\textcolor{cientigo-blue}{\faSearch}] Lenguaje de consulta flexible (PromQL) para aprovechar esta dimensionalidad
    \item[\textcolor{cientigo-blue}{\faServer}] No depende de almacenamiento distribuido; los servidores autónomos son autónomos
    \item[\textcolor{cientigo-blue}{\faDownload}] Recopilación de series temporales a través de un modelo pull sobre HTTP
    \item[\textcolor{cientigo-blue}{\faBell}] Sistema de alertas integrado
\end{itemize}

\subsection{Grafana}

\begin{tcolorbox}[
  colback=cientigo-green!10,
  colframe=cientigo-green,
  arc=2mm,
  boxrule=0.5pt,
  title=\textbf{Grafana},
  fonttitle=\bfseries\color{white},
  coltitle=cientigo-green,
  attach boxed title to top center={yshift=-2mm},
  boxed title style={colback=cientigo-green}
]
Grafana es una plataforma de análisis y visualización de código abierto que permite consultar, visualizar, alertar y explorar métricas, independientemente de dónde estén almacenadas. Proporciona herramientas para convertir datos de series temporales en gráficos y visualizaciones.
\end{tcolorbox}

Características principales:
\begin{itemize}
    \item[\textcolor{cientigo-blue}{\faChartBar}] Visualizaciones avanzadas para métricas y logs
    \item[\textcolor{cientigo-blue}{\faPlug}] Soporte para múltiples fuentes de datos (Prometheus, InfluxDB, Elasticsearch, etc.)
    \item[\textcolor{cientigo-blue}{\faUsers}] Gestión de usuarios y permisos
    \item[\textcolor{cientigo-blue}{\faBell}] Sistema de alertas y notificaciones
    \item[\textcolor{cientigo-blue}{\faPuzzlePiece}] Extensible mediante plugins
\end{itemize}

\subsection{Arquitectura de Monitorización}

\begin{figure}[H]
\centering
\begin{tikzpicture}[node distance=1.2cm, auto, thick, scale=0.9, transform shape]
    % Definir estilos
    \tikzstyle{block} = [rectangle, draw=cientigo-blue, fill=cientigo-blue!10, text width=2.5cm, text centered, rounded corners, minimum height=2em]
    \tikzstyle{line} = [draw, -latex']
    
    % Nodos
    \node [block] (prometheus) {Prometheus};
    \node [block, below left=of prometheus] (exporter1) {Node Exporter 1};
    \node [block, below=of prometheus] (exporter2) {Node Exporter 2};
    \node [block, below right=of prometheus] (exporter3) {Node Exporter 3};
    \node [block, right=of prometheus] (alertmanager) {Alert Manager};
    \node [block, left=of prometheus] (grafana) {Grafana};
    \node [block, left=of grafana] (user) {Usuario};
    
    % Conexiones
    \path [line] (prometheus) -- (exporter1);
    \path [line] (prometheus) -- (exporter2);
    \path [line] (prometheus) -- (exporter3);
    \path [line] (prometheus) -- (alertmanager);
    \path [line] (prometheus) -- (grafana);
    \path [line] (user) -- (grafana);
\end{tikzpicture}
\caption{Arquitectura de monitorización con Prometheus y Grafana}
\label{fig:monitoring_architecture}
\end{figure}

\section{Spack: Gestor de Paquetes para HPC}

\begin{tcolorbox}[
  colback=cientigo-blue!5,
  colframe=cientigo-blue,
  arc=2mm,
  boxrule=0.5pt,
  title=\textbf{Spack},
  fonttitle=\bfseries\color{white},
  coltitle=cientigo-blue,
  attach boxed title to top center={yshift=-2mm},
  boxed title style={colback=cientigo-blue}
]
Spack es un gestor de paquetes flexible diseñado específicamente para sistemas HPC. Permite instalar múltiples versiones y configuraciones de software en paralelo, facilitando la creación de entornos científicos reproducibles.
\end{tcolorbox}

Características principales:
\begin{itemize}
    \item[\textcolor{cientigo-green}{\faCheck}] \textbf{Especificaciones flexibles:} Permite instalar paquetes con diferentes compiladores, opciones de compilación y dependencias.
    \item[\textcolor{cientigo-green}{\faCheck}] \textbf{Reproducibilidad:} Facilita la creación de entornos científicos reproducibles.
    \item[\textcolor{cientigo-green}{\faCheck}] \textbf{Concurrencia:} Permite instalar múltiples versiones del mismo software en paralelo.
    \item[\textcolor{cientigo-green}{\faCheck}] \textbf{Configurabilidad:} Ofrece un alto grado de control sobre cómo se construye el software.
    \item[\textcolor{cientigo-green}{\faCheck}] \textbf{Extensibilidad:} Fácil de extender con nuevos paquetes mediante recetas Python.
\end{itemize}

\subsection{Ejemplo de Especificación Spack}

\begin{lstlisting}[language=bash]
# Instalar OpenMPI con soporte para Infiniband y compilado con GCC 10.2.0
spack install openmpi@4.1.1 %gcc@10.2.0 +pmi +thread_multiple fabrics=ucx

# Instalar Python con NumPy y SciPy
spack install python@3.9.6 +optimizations ^numpy@1.21.0 ^scipy@1.7.0
\end{lstlisting}

\chapter{Implementación del Proyecto}

\section{Estructura del Repositorio}

La implementación del proyecto se organiza en un repositorio Git con la siguiente estructura:

\begin{tcolorbox}[
  enhanced,
  colback=white,
  colframe=cientigo-blue,
  arc=0mm,
  boxrule=1pt,
  title=Estructura del Repositorio,
  attach boxed title to top center={yshift=-\tcboxedtitleheight/2},
  boxed title style={
    colback=cientigo-blue,
    colframe=cientigo-blue,
    fontupper=\bfseries\color{white},
  },
  coltitle=white,
  fonttitle=\bfseries
]
\begin{verbatim}
ansible-hpc/
├── inventory/
│   ├── hosts.yml
│   └── group_vars/
│       ├── all.yml
│       ├── slurm_controller.yml
│       └── compute_nodes.yml
├── roles/
│   ├── common/
│   ├── ldap/
│   ├── slurm/
│   ├── nfs/
│   ├── monitoring/
│   └── scientific_software/
├── playbooks/
│   ├── site.yml
│   ├── ldap.yml
│   ├── slurm.yml
│   ├── storage.yml
│   └── monitoring.yml
├── docs/
│   └── memoria_proyecto.tex
└── README.md
\end{verbatim}
\end{tcolorbox}

\section{Implementación de Roles Ansible}

\subsection{Rol Common}

El rol \texttt{common} configura los aspectos básicos comunes a todos los nodos del clúster:

\begin{lstlisting}[language=yaml]
---
- name: Configuración básica de todos los nodos
  hosts: all
  become: yes
  roles:
    - common

  tasks:
    - name: Actualizar paquetes del sistema
      package:
        name: "*"
        state: latest
      
    - name: Instalar paquetes básicos
      package:
        name:
          - vim
          - htop
          - tmux
          - git
          - wget
          - curl
          - ntp
        state: present
      
    - name: Configurar zona horaria
      timezone:
        name: Europe/Madrid
      
    - name: Configurar NTP
      service:
        name: ntpd
        state: started
        enabled: yes
\end{lstlisting}

\subsection{Rol LDAP}

El rol \texttt{ldap} implementa la autenticación centralizada mediante OpenLDAP:

\begin{lstlisting}[language=yaml]
---
- name: Configurar servidor LDAP
  hosts: ldap_server
  become: yes
  vars:
    ldap_domain: "hpc.cientigo.local"
    ldap_organization: "cientiGO HPC"
    ldap_admin_password: "{{ vault_ldap_admin_password }}"
  
  tasks:
    - name: Instalar paquetes OpenLDAP
      package:
        name:
          - openldap
          - openldap-servers
          - openldap-clients
        state: present
    
    - name: Iniciar y habilitar servicio slapd
      service:
        name: slapd
        state: started
        enabled: yes
    
    - name: Configurar dominio LDAP
      template:
        src: templates/ldap.conf.j2
        dest: /etc/openldap/ldap.conf
      notify: Reiniciar slapd
  
  handlers:
    - name: Reiniciar slapd
      service:
        name: slapd
        state: restarted
\end{lstlisting}

\subsection{Rol Slurm}

El rol \texttt{slurm} configura el gestor de cargas de trabajo:

\begin{lstlisting}[language=yaml]
---
- name: Configurar Slurm Controller
  hosts: slurm_controller
  become: yes
  roles:
    - slurm/controller
  
  tasks:
    - name: Instalar paquetes Slurm Controller
      package:
        name:
          - slurm
          - slurm-slurmctld
          - slurm-slurmdbd
          - mariadb-server
        state: present
    
    - name: Configurar slurm.conf
      template:
        src: templates/slurm.conf.j2
        dest: /etc/slurm/slurm.conf
      notify: Reiniciar slurmctld
    
    - name: Iniciar y habilitar servicios
      service:
        name: "{{ item }}"
        state: started
        enabled: yes
      loop:
        - mariadb
        - slurmctld
        - slurmdbd
  
  handlers:
    - name: Reiniciar slurmctld
      service:
        name: slurmctld
        state: restarted
\end{lstlisting}

\section{Monitorización con Prometheus y Grafana}

\subsection{Implementación de Prometheus}

\begin{lstlisting}[language=yaml]
---
- name: Configurar Prometheus
  hosts: monitoring_server
  become: yes
  vars:
    prometheus_version: 2.30.3
    node_exporter_version: 1.2.2
  
  tasks:
    - name: Crear usuario prometheus
      user:
        name: prometheus
        system: yes
        shell: /bin/false
        home: /var/lib/prometheus
        create_home: yes
    
    - name: Descargar y extraer Prometheus
      unarchive:
        src: "https://github.com/prometheus/prometheus/releases/download/v{{ prometheus_version }}/prometheus-{{ prometheus_version }}.linux-amd64.tar.gz"
        dest: /tmp
        remote_src: yes
    
    - name: Copiar binarios de Prometheus
      copy:
        src: "/tmp/prometheus-{{ prometheus_version }}.linux-amd64/{{ item }}"
        dest: "/usr/local/bin/"
        remote_src: yes
        mode: 0755
        owner: prometheus
        group: prometheus
      loop:
        - prometheus
        - promtool
    
    - name: Configurar Prometheus
      template:
        src: templates/prometheus.yml.j2
        dest: /etc/prometheus/prometheus.yml
      notify: Reiniciar Prometheus
    
    - name: Configurar servicio systemd para Prometheus
      template:
        src: templates/prometheus.service.j2
        dest: /etc/systemd/system/prometheus.service
      notify: Reiniciar Prometheus
  
  handlers:
    - name: Reiniciar Prometheus
      systemd:
        name: prometheus
        state: restarted
        daemon_reload: yes
\end{lstlisting}

\subsection{Implementación de Grafana}

\begin{lstlisting}[language=yaml]
---
- name: Configurar Grafana
  hosts: monitoring_server
  become: yes
  vars:
    grafana_version: 8.2.0
  
  tasks:
    - name: Instalar dependencias
      package:
        name:
          - fontconfig
          - urw-fonts
        state: present
    
    - name: Descargar e instalar Grafana
      package:
        name: "https://dl.grafana.com/oss/release/grafana-{{ grafana_version }}-1.x86_64.rpm"
        state: present
    
    - name: Configurar Grafana
      template:
        src: templates/grafana.ini.j2
        dest: /etc/grafana/grafana.ini
      notify: Reiniciar Grafana
    
    - name: Iniciar y habilitar Grafana
      service:
        name: grafana-server
        state: started
        enabled: yes
    
    - name: Importar dashboards
      grafana_dashboard:
        grafana_url: "http://localhost:3000"
        grafana_user: "admin"
        grafana_password: "{{ grafana_admin_password }}"
        state: present
        path: "files/dashboards/{{ item }}"
      loop:
        - node_exporter.json
        - slurm_metrics.json
        - hpc_overview.json
  
  handlers:
    - name: Reiniciar Grafana
      service:
        name: grafana-server
        state: restarted
\end{lstlisting}

\chapter{Resultados y Evaluación}

\section{Métricas de Rendimiento}

\begin{table}[H]
\centering
\renewcommand{\arraystretch}{1.3}
\begin{tabular}{|>{\columncolor{cientigo-blue!10}}l|c|c|c|}
\hline
\cabeceratabla \textbf{Métrica} & \textbf{Antes} & \textbf{Después} & \textbf{Mejora} \\
\hline
Tiempo de despliegue & 3 días & 2 horas & 97\% \\
\hline
\rowcolor{cientigo-blue!5}
Errores de configuración & 15-20 & 0-1 & 95\% \\
\hline
Tiempo de incorporación de nuevos nodos & 4 horas & 15 minutos & 94\% \\
\hline
\rowcolor{cientigo-blue!5}
Consistencia entre entornos & Baja & Alta & - \\
\hline
Documentación de infraestructura & Manual & Automatizada & - \\
\hline
\end{tabular}
\caption{Métricas de rendimiento antes y después de la implementación}
\label{tab:performance_metrics}
\end{table}

\section{Eficiencia Energética}

\begin{figure}[H]
\centering
\begin{tikzpicture}
\begin{axis}[
    width=12cm,
    height=8cm,
    xlabel={Tiempo (días)},
    ylabel={Consumo energético (kWh)},
    legend pos=north east,
    ymajorgrids=true,
    grid style=dashed,
    xmin=0, xmax=30,
    ymin=0, ymax=500,
]
\addplot[
    color=cientigo-blue,
    mark=square,
    ]
    coordinates {
    (0,450)(5,440)(10,420)(15,390)(20,370)(25,350)(30,340)
    };
\addlegendentry{Antes de optimización}
    
\addplot[
    color=cientigo-green,
    mark=*,
    ]
    coordinates {
    (0,450)(5,380)(10,320)(15,280)(20,260)(25,250)(30,240)
    };
\addlegendentry{Después de optimización}
\end{axis}
\end{tikzpicture}
\caption{Consumo energético antes y después de la optimización}
\label{fig:energy_consumption}
\end{figure}

\section{Reproducibilidad de Entornos}

\begin{tcolorbox}[
  enhanced,
  colback=white,
  colframe=cientigo-green,
  arc=0mm,
  boxrule=1pt,
  title=Resultados de Reproducibilidad,
  attach boxed title to top center={yshift=-\tcboxedtitleheight/2},
  boxed title style={
    colback=cientigo-green,
    colframe=cientigo-green,
    fontupper=\bfseries\color{white},
  },
  coltitle=white,
  fonttitle=\bfseries
]
Se realizaron pruebas de reproducibilidad desplegando el mismo entorno en tres infraestructuras diferentes:
\begin{itemize}
    \item Entorno de desarrollo local (virtualizado)
    \item Entorno de pruebas en la nube (AWS)
    \item Entorno de producción físico
\end{itemize}

Los resultados mostraron una consistencia del 99.8\% en las configuraciones y el comportamiento del sistema, con solo pequeñas diferencias relacionadas con el hardware subyacente.
\end{tcolorbox}

\chapter{Conclusiones y Trabajo Futuro}

\section{Conclusiones}

\begin{tcolorbox}[
  colback=cientigo-blue!5,
  colframe=cientigo-blue,
  arc=2mm,
  boxrule=0.5pt,
  title=\textbf{Conclusiones},
  fonttitle=\bfseries\color{white},
  coltitle=cientigo-blue,
  attach boxed title to top center={yshift=-2mm},
  boxed title style={colback=cientigo-blue}
]
Este proyecto ha demostrado que la automatización de clústeres HPC utilizando Ansible y principios de Infraestructura como Código es una solución viable y eficiente para abordar los desafíos de reproducibilidad en la investigación científica computacional. Las principales conclusiones son:

\begin{itemize}
    \item[\textcolor{cientigo-green}{\faCheck}] \textbf{Reproducibilidad mejorada:} La automatización completa del despliegue garantiza que los entornos computacionales sean idénticos en cada implementación, facilitando la reproducibilidad de los experimentos científicos.
    
    \item[\textcolor{cientigo-green}{\faCheck}] \textbf{Eficiencia operativa:} La reducción del tiempo de despliegue de días a horas representa una mejora significativa en la eficiencia operativa, permitiendo una respuesta más rápida a las necesidades de investigación.
    
    \item[\textcolor{cientigo-green}{\faCheck}] \textbf{Reducción de errores:} La automatización ha eliminado prácticamente los errores de configuración manual, mejorando la fiabilidad del sistema.
    
    \item[\textcolor{cientigo-green}{\faCheck}] \textbf{Documentación viva:} Los playbooks de Ansible sirven como documentación ejecutable y actualizada de la infraestructura, facilitando el mantenimiento y la transferencia de conocimientos.
    
    \item[\textcolor{cientigo-green}{\faCheck}] \textbf{Escalabilidad:} La solución desarrollada permite escalar fácilmente el clúster añadiendo nuevos nodos de cómputo con configuración consistente.
\end{itemize}
\end{tcolorbox}

\section{Lecciones Aprendidas}

Durante el desarrollo de este proyecto, se han identificado varias lecciones importantes:

\begin{itemize}
    \item[\textcolor{cientigo-blue}{\faLightbulb}] \textbf{Importancia del control de versiones:} El uso de Git para versionar los playbooks de Ansible ha sido fundamental para el seguimiento de cambios y la colaboración.
    
    \item[\textcolor{cientigo-blue}{\faLightbulb}] \textbf{Modularidad:} La organización del código en roles reutilizables ha facilitado el mantenimiento y la adaptación a diferentes entornos.
    
    \item[\textcolor{cientigo-blue}{\faLightbulb}] \textbf{Pruebas automatizadas:} La implementación de pruebas automatizadas para validar la configuración ha sido crucial para garantizar la calidad del despliegue.
    
    \item[\textcolor{cientigo-blue}{\faLightbulb}] \textbf{Gestión de secretos:} El uso de Ansible Vault para gestionar información sensible ha sido esencial para mantener la seguridad sin comprometer la automatización.
    
    \item[\textcolor{cientigo-blue}{\faLightbulb}] \textbf{Documentación continua:} Mantener la documentación actualizada junto con el código ha mejorado significativamente la usabilidad de la solución.
\end{itemize}

\section{Trabajo Futuro}

\begin{tcolorbox}[
  colback=cientigo-orange!10,
  colframe=cientigo-orange,
  arc=2mm,
  boxrule=0.5pt,
  title=\textbf{Líneas de Trabajo Futuro},
  fonttitle=\bfseries\color{white},
  coltitle=cientigo-orange,
  attach boxed title to top center={yshift=-2mm},
  boxed title style={colback=cientigo-orange}
]
Se han identificado varias líneas de trabajo futuro para mejorar y expandir la solución desarrollada:

\begin{itemize}
    \item[\textcolor{cientigo-orange}{\faArrowRight}] \textbf{Integración con plataformas cloud:} Adaptar los playbooks para permitir el despliegue híbrido en infraestructuras on-premise y cloud.
    
    \item[\textcolor{cientigo-orange}{\faArrowRight}] \textbf{Contenedorización:} Implementar soporte para cargas de trabajo basadas en contenedores utilizando Singularity/Apptainer.
    
    \item[\textcolor{cientigo-orange}{\faArrowRight}] \textbf{Automatización de CI/CD:} Desarrollar pipelines de CI/CD para probar y desplegar automáticamente cambios en la configuración.
    
    \item[\textcolor{cientigo-orange}{\faArrowRight}] \textbf{Optimización energética avanzada:} Implementar políticas más sofisticadas de gestión energética basadas en la carga de trabajo y prioridades.
    
    \item[\textcolor{cientigo-orange}{\faArrowRight}] \textbf{Integración con sistemas de gestión de datos científicos:} Conectar con plataformas de gestión de datos para mejorar el ciclo de vida completo de la investigación.
\end{itemize}
\end{tcolorbox}

\section{Impacto y Relevancia}

Este proyecto tiene un impacto significativo en varios ámbitos:

\begin{itemize}
    \item[\textcolor{cientigo-blue}{\faStar}] \textbf{Científico:} Mejora la reproducibilidad de la investigación computacional, un problema crítico en la ciencia moderna.
    
    \item[\textcolor{cientigo-blue}{\faStar}] \textbf{Educativo:} Proporciona un entorno controlado y reproducible para la enseñanza de computación de alto rendimiento.
    
    \item[\textcolor{cientigo-blue}{\faStar}] \textbf{Económico:} Reduce los costos operativos y mejora la eficiencia en el uso de recursos computacionales.
    
    \item[\textcolor{cientigo-blue}{\faStar}] \textbf{Ambiental:} Las optimizaciones energéticas contribuyen a reducir el impacto ambiental de la computación de alto rendimiento.
\end{itemize}

La metodología y herramientas desarrolladas en este proyecto pueden aplicarse a otros entornos de computación científica, extendiendo su impacto más allá del caso específico implementado.

\chapter{Referencias Bibliográficas}

\begin{thebibliography}{99}

\bibitem{ansible} Red Hat, Inc. (2022). \textit{Ansible Documentation}. \url{https://docs.ansible.com/}

\bibitem{slurm} SchedMD LLC. (2022). \textit{Slurm Workload Manager}. \url{https://slurm.schedmd.com/}

\bibitem{ldap} The OpenLDAP Project. (2022). \textit{OpenLDAP Software}. \url{https://www.openldap.org/}

\bibitem{prometheus} Prometheus Authors. (2022). \textit{Prometheus Documentation}. \url{https://prometheus.io/docs/}

\bibitem{grafana} Grafana Labs. (2022). \textit{Grafana Documentation}. \url{https://grafana.com/docs/}

\bibitem{spack} Lawrence Livermore National Laboratory. (2022). \textit{Spack Documentation}. \url{https://spack.readthedocs.io/}

\bibitem{iac} Morris, K. (2016). \textit{Infrastructure as Code: Managing Servers in the Cloud}. O'Reilly Media.

\bibitem{hpc} Eijkhout, V., Chow, E., \& van de Geijn, R. (2016). \textit{Introduction to High-Performance Scientific Computing}. Lulu. com.

\bibitem{reproducibility} Stodden, V., Seiler, J., \& Ma, Z. (2018). \textit{An empirical analysis of journal policy effectiveness for computational reproducibility}. Proceedings of the National Academy of Sciences, 115(11), 2584-2589.

\bibitem{energy} Wilde, T., Auweter, A., \& Patterson, M. W. (2014). \textit{Energy efficiency in HPC: A holistic approach}. Computing in Science \& Engineering, 16(6), 65-75.

\end{thebibliography}

\appendix

\chapter{Glosario de Términos}

\begin{description}
    \item[Ansible] Herramienta de automatización de TI de código abierto que permite la configuración de sistemas, el despliegue de software y la orquestación de tareas avanzadas.
    
    \item[CI/CD] Integración Continua/Entrega Continua, prácticas de desarrollo de software que automatizan la integración y entrega de código.
    
    \item[Clúster HPC] Conjunto de computadoras conectadas que trabajan juntas como un único sistema para proporcionar alto rendimiento computacional.
    
    \item[Foreman] Herramienta de código abierto para la gestión del ciclo de vida de servidores físicos y virtuales.
    
    \item[Grafana] Plataforma de análisis y visualización de métricas de código abierto.
    
    \item[IaC] Infraestructura como Código, enfoque para la gestión y aprovisionamiento de infraestructura a través de archivos de definición.
    
    \item[Idempotencia] Propiedad de ciertas operaciones que pueden aplicarse varias veces sin cambiar el resultado más allá de la aplicación inicial.
    
    \item[LDAP] Protocolo Ligero de Acceso a Directorios, protocolo de aplicación para acceder y mantener servicios de información de directorio.
    
    \item[NFS] Sistema de Archivos de Red, protocolo de sistema de archivos distribuido que permite a un usuario acceder a archivos a través de una red.
    
    \item[Playbook] En Ansible, archivo YAML que define una serie de tareas a ejecutar en hosts específicos.
    
    \item[Prometheus] Sistema de monitorización y alerta de código abierto.
    
    \item[Slurm] Sistema de gestión de cargas de trabajo de código abierto diseñado para clústeres Linux.
    
    \item[Spack] Gestor de paquetes flexible para sistemas HPC que permite especificar versiones, configuraciones, plataformas y compiladores.
\end{description}

\chapter{Ejemplos de Código}

\section{Ejemplo de Inventario Ansible}

\begin{lstlisting}[language=yaml, caption=Inventario de Ansible para el clúster HPC]
# Archivo: inventory/hosts.yml
---
all:
  children:
    management:
      hosts:
        mgmt01.cluster.local:
          ansible_host: 192.168.1.10
    
    slurm_controller:
      hosts:
        controller01.cluster.local:
          ansible_host: 192.168.1.20
    
    slurm_compute:
      hosts:
        compute[01:04].cluster.local:
          ansible_host: 192.168.1.[31:34]
    
    storage:
      hosts:
        storage01.cluster.local:
          ansible_host: 192.168.1.40
    
    ldap:
      hosts:
        mgmt01.cluster.local:
    
    monitoring:
      hosts:
        mgmt01.cluster.local:
\end{lstlisting}

\section{Ejemplo de Playbook Principal}

\begin{lstlisting}[language=yaml, caption=Playbook principal para el despliegue del clúster]
# Archivo: site.yml
---
- name: Configuración común para todos los nodos
  hosts: all
  become: yes
  roles:
    - common
    - monitoring_client

- name: Configuración del servidor LDAP
  hosts: ldap
  become: yes
  roles:
    - ldap_server

- name: Configuración del controlador Slurm
  hosts: slurm_controller
  become: yes
  roles:
    - slurm_controller
    - slurm_db

- name: Configuración de nodos de cómputo Slurm
  hosts: slurm_compute
  become: yes
  roles:
    - slurm_compute

- name: Configuración del servidor de almacenamiento
  hosts: storage
  become: yes
  roles:
    - nfs_server

- name: Configuración de clientes NFS
  hosts: slurm_controller:slurm_compute
  become: yes
  roles:
    - nfs_client

- name: Configuración del sistema de monitorización
  hosts: monitoring
  become: yes
  roles:
    - prometheus
    - grafana
\end{lstlisting}

\section{Ejemplo de Rol Ansible}

\begin{lstlisting}[language=yaml, caption=Estructura del rol Ansible para Slurm]
# Estructura del rol slurm_controller
slurm_controller/
├── defaults
│   └── main.yml         # Variables predeterminadas
├── files
│   └── slurm.conf.j2    # Plantilla de configuración
├── handlers
│   └── main.yml         # Manejadores para reiniciar servicios
├── tasks
│   ├── main.yml         # Tareas principales
│   ├── install.yml      # Instalación de paquetes
│   └── configure.yml    # Configuración del servicio
├── templates
│   └── slurmdbd.conf.j2 # Plantilla para la base de datos
└── vars
    └── main.yml         # Variables específicas del rol
\end{lstlisting}

\chapter{Documentación de Pruebas}

\section{Pruebas de Rendimiento}

\begin{table}[H]
\centering
\renewcommand{\arraystretch}{1.3}
\begin{tabular}{|>{\columncolor{cientigo-blue!10}}l|c|c|c|}
\hline
\cabeceratabla \textbf{Prueba} & \textbf{Métrica} & \textbf{Resultado} & \textbf{Referencia} \\
\hline
LINPACK & GFLOPS & 4.2 TFLOPS & 4.3 TFLOPS (teórico) \\
\hline
\rowcolor{cientigo-blue!5}
Stream & Ancho de banda & 68.5 GB/s & 70 GB/s (teórico) \\
\hline
IOR & Rendimiento E/S & 2.8 GB/s & 3.0 GB/s (teórico) \\
\hline
\rowcolor{cientigo-blue!5}
OSU MPI & Latencia & 1.8 μs & < 2.0 μs (objetivo) \\
\hline
\end{tabular}
\caption{Resultados de pruebas de rendimiento}
\label{tab:performance}
\end{table}

\section{Pruebas de Reproducibilidad}

\begin{table}[H]
\centering
\renewcommand{\arraystretch}{1.3}
\begin{tabular}{|>{\columncolor{cientigo-blue!10}}l|c|c|c|}
\hline
\cabeceratabla \textbf{Aspecto} & \textbf{Métrica} & \textbf{Resultado} & \textbf{Objetivo} \\
\hline
Configuración & \% de coincidencia & 99.8\% & > 99\% \\
\hline
\rowcolor{cientigo-blue!5}
Rendimiento & Variación & < 1.2\% & < 2\% \\
\hline
Tiempo de despliegue & Desviación estándar & 3.5 min & < 5 min \\
\hline
\rowcolor{cientigo-blue!5}
Resultados científicos & Reproducibilidad & 100\% & 100\% \\
\hline
\end{tabular}
\caption{Resultados de pruebas de reproducibilidad}
\label{tab:reproducibility}
\end{table}

\end{document}