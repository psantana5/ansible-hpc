% Capítulo 2: Planificación del Proyecto
\chapter{Planificación del Proyecto}
\section{Cronograma}
\subsection{Fases del Proyecto}
El proyecto se divide en las siguientes fases principales:

\begin{infocaja}
\textbf{Fase 1: Preparación y Análisis (Semanas 1-2)}
\begin{itemize}
    \item Análisis de requisitos y definición de alcance
    \item Evaluación de infraestructura existente
    \item Diseño de arquitectura del clúster
    \item Selección de tecnologías y herramientas
\end{itemize}
\end{infocaja}

\begin{infocaja}
\textbf{Fase 2: Desarrollo de Automatización (Semanas 3-6)}
\begin{itemize}
    \item Desarrollo de roles de Ansible
    \item Implementación de playbooks
    \item Configuración de integración continua
    \item Pruebas de automatización
\end{itemize}
\end{infocaja}

\begin{infocaja}
\textbf{Fase 3: Implementación (Semanas 7-10)}
\begin{itemize}
    \item Despliegue de infraestructura base
    \item Configuración de servicios core
    \item Implementación de seguridad
    \item Integración de sistemas de monitoreo
\end{itemize}
\end{infocaja}

\begin{infocaja}
\textbf{Fase 4: Pruebas y Optimización (Semanas 11-12)}
\begin{itemize}
    \item Pruebas de rendimiento
    \item Optimización de configuración
    \item Validación de seguridad
    \item Documentación técnica
\end{itemize}
\end{infocaja}

\section{Asignación de Recursos}
\subsection{Recursos Humanos}
\begin{itemize}
    \item \textbf{Equipo de Desarrollo}
    \begin{itemize}
        \item 2 Ingenieros DevOps
        \item 1 Especialista en HPC
        \item 1 Administrador de Sistemas
    \end{itemize}
    \item \textbf{Equipo de Operaciones}
    \begin{itemize}
        \item 1 Ingeniero de Redes
        \item 1 Especialista en Seguridad
        \item 1 Técnico de Soporte
    \end{itemize}
\end{itemize}

\subsection{Recursos Técnicos}
\begin{itemize}
    \item \textbf{Infraestructura de Desarrollo}
    \begin{itemize}
        \item Servidor de CI/CD
        \item Entorno de pruebas
        \item Herramientas de desarrollo
    \end{itemize}
    \item \textbf{Infraestructura de Producción}
    \begin{itemize}
        \item Nodos de cómputo
        \item Servidores de almacenamiento
        \item Equipos de red
    \end{itemize}
\end{itemize}

\section{Gestión de Riesgos}
\subsection{Identificación de Riesgos}
\begin{longtable}{|p{4cm}|p{4cm}|p{4cm}|}
\hline
\cabeceratabla Riesgo & Impacto & Mitigación \\
\hline
Fallo en la automatización & Alto & Pruebas exhaustivas y rollback planificado \\
\hline
Problemas de rendimiento & Medio & Benchmarking continuo y optimización \\
\hline
Vulnerabilidades de seguridad & Alto & Auditorías regulares y actualizaciones \\
\hline
Escasez de recursos & Medio & Planificación de capacidad y escalado \\
\hline
\end{longtable}

\section{Aseguramiento de Calidad}
\subsection{Procesos de QA}
\begin{itemize}
    \item \textbf{Pruebas Automatizadas}
    \begin{itemize}
        \item Pruebas unitarias de roles Ansible
        \item Pruebas de integración
        \item Pruebas de rendimiento
    \end{itemize}
    \item \textbf{Revisiones de Código}
    \begin{itemize}
        \item Revisión por pares
        \item Análisis estático de código
        \item Cumplimiento de estándares
    \end{itemize}
\end{itemize}

\section{Plan de Comunicación}
\subsection{Canales de Comunicación}
\begin{itemize}
    \item \textbf{Internos}
    \begin{itemize}
        \item Reuniones diarias de equipo
        \item Documentación técnica
        \item Sistema de tickets
    \end{itemize}
    \item \textbf{Externos}
    \begin{itemize}
        \item Informes de progreso
        \item Documentación de usuario
        \item Soporte técnico
    \end{itemize}
\end{itemize}

% Capítulo 3: Antecedentes Técnicos
\chapter{Antecedentes Técnicos}
\section{Computación de Alto Rendimiento (HPC)}
\subsection{Conceptos Fundamentales}
La computación de alto rendimiento (HPC) se refiere a la práctica de agregar potencia de cómputo para resolver problemas complejos en ciencia, ingeniería o negocios. Los clústeres HPC modernos típicamente consisten en:

\begin{itemize}
    \item \textbf{Nodos de Cómputo}: Servidores optimizados para procesamiento paralelo
    \item \textbf{Nodos de Acceso}: Puntos de entrada para usuarios
    \item \textbf{Red de Alta Velocidad}: Interconexión de nodos
    \item \textbf{Sistema de Almacenamiento}: Almacenamiento compartido de alto rendimiento
    \item \textbf{Gestor de Colas}: Software para programación de trabajos
\end{itemize}

\section{Ansible}
\subsection{Fundamentos de Automatización}
Ansible es una plataforma de automatización de TI que utiliza un lenguaje declarativo para describir la configuración del sistema. Sus características clave incluyen:

\begin{infocaja}
\textbf{Ventajas de Ansible para HPC}
\begin{itemize}
    \item \textbf{Idempotencia}: Las tareas se pueden ejecutar múltiples veces sin efectos secundarios
    \item \textbf{Agente-less}: No requiere software adicional en los nodos gestionados
    \item \textbf{YAML}: Sintaxis simple y legible
    \item \textbf{Modularidad}: Roles y playbooks reutilizables
\end{itemize}
\end{infocaja}

\section{Gestor de Cargas de Trabajo (SLURM)}
\subsection{Arquitectura y Componentes}
SLURM (Simple Linux Utility for Resource Management) es un gestor de recursos y planificador de trabajos de código abierto. Sus componentes principales son:

\begin{itemize}
    \item \textbf{slurmctld}: Demonio de control central
    \item \textbf{slurmd}: Demonio de nodo de cómputo
    \item \textbf{slurmdbd}: Demonio de base de datos
    \item \textbf{sacctmgr}: Herramienta de gestión de cuentas
\end{itemize}

\section{Tecnologías de Almacenamiento}
\subsection{Sistemas de Archivos Distribuidos}
\begin{infocaja}
\textbf{Componentes de Almacenamiento}
\begin{itemize}
    \item \textbf{NFS}: Sistema de archivos en red para datos compartidos
    \item \textbf{Lustre}: Sistema de archivos paralelo para alto rendimiento
    \item \textbf{GPFS}: Sistema de archivos paralelo de IBM
    \item \textbf{Ceph}: Sistema de almacenamiento distribuido
\end{itemize}
\end{infocaja}

\section{Autenticación y Seguridad}
\subsection{Infraestructura de Identidad}
\begin{itemize}
    \item \textbf{OpenLDAP}
    \begin{itemize}
        \item Directorio centralizado de usuarios
        \item Autenticación unificada
        \item Gestión de grupos y permisos
    \end{itemize}
    \item \textbf{Seguridad}
    \begin{itemize}
        \item Firewalls y políticas de red
        \item Cifrado de datos en tránsito
        \item Control de acceso basado en roles
    \end{itemize}
\end{itemize}

\section{Monitoreo y Métricas}
\subsection{Stack de Monitoreo}
\begin{infocaja}
\textbf{Componentes de Monitoreo}
\begin{itemize}
    \item \textbf{Prometheus}: Sistema de monitoreo y alertas
    \item \textbf{Grafana}: Visualización de métricas
    \item \textbf{Node Exporter}: Métricas del sistema
    \item \textbf{Alertmanager}: Gestión de alertas
\end{itemize}
\end{infocaja}

% Capítulo 4: Análisis de Costos
\chapter{Análisis de Costos}
\section{Costos de Licencias de Software}
\subsection{Software Propietario vs Open Source}
\begin{longtable}{|p{4cm}|p{4cm}|p{4cm}|}
\hline
\cabeceratabla Componente & Tipo & Costo \\
\hline
Sistema Operativo & Open Source & \$0 \\
\hline
Ansible & Open Source & \$0 \\
\hline
SLURM & Open Source & \$0 \\
\hline
Monitoreo & Open Source & \$0 \\
\hline
\end{longtable}

\section{Costos de Hardware}
\subsection{Infraestructura Base}
\begin{itemize}
    \item \textbf{Nodos de Cómputo}
    \begin{itemize}
        \item 10 nodos de cómputo: \$150,000
        \item 2 nodos de acceso: \$20,000
        \item 1 nodo de gestión: \$15,000
    \end{itemize}
    \item \textbf{Almacenamiento}
    \begin{itemize}
        \item Sistema de almacenamiento compartido: \$80,000
        \item Sistema de backup: \$30,000
    \end{itemize}
    \item \textbf{Red}
    \begin{itemize}
        \item Switches de alta velocidad: \$40,000
        \item Cables y conectores: \$5,000
    \end{itemize}
\end{itemize}

\section{Costos de Implementación}
\subsection{Recursos Humanos}
\begin{itemize}
    \item \textbf{Equipo de Desarrollo}
    \begin{itemize}
        \item 2 Ingenieros DevOps (3 meses): \$60,000
        \item 1 Especialista HPC (3 meses): \$45,000
    \end{itemize}
    \item \textbf{Equipo de Operaciones}
    \begin{itemize}
        \item 1 Ingeniero de Redes (2 meses): \$20,000
        \item 1 Especialista en Seguridad (2 meses): \$20,000
    \end{itemize}
\end{itemize}

\section{Costos Operativos}
\subsection{Gastos Recurrentes}
\begin{itemize}
    \item \textbf{Personal}
    \begin{itemize}
        \item Administrador del clúster: \$80,000/año
        \item Soporte técnico: \$60,000/año
    \end{itemize}
    \item \textbf{Infraestructura}
    \begin{itemize}
        \item Energía y refrigeración: \$30,000/año
        \item Mantenimiento: \$20,000/año
    \end{itemize}
\end{itemize}

\section{Costo Total de Propiedad (TCO)}
\subsection{Análisis a 5 Años}
\begin{infocaja}
\textbf{Desglose del TCO}
\begin{itemize}
    \item \textbf{Costos Iniciales}: \$420,000
    \item \textbf{Costos Operativos Anuales}: \$190,000
    \item \textbf{TCO a 5 años}: \$1,370,000
\end{itemize}
\end{infocaja}

% Capítulo 5: Implementación Técnica
\chapter{Implementación Técnica}
\section{Estructura de Roles de Ansible}
\subsection{Organización del Código}
\begin{lstlisting}[language=yaml]
ansible-hpc/
├── inventory/
│   ├── production/
│   └── staging/
├── group_vars/
│   ├── all/
│   ├── compute/
│   └── login/
├── roles/
│   ├── common/
│   ├── slurm/
│   ├── storage/
│   ├── monitoring/
│   └── security/
└── playbooks/
    ├── site.yml
    ├── compute.yml
    └── monitoring.yml
\end{lstlisting}

\section{Detalles de la Implementación}
\subsection{Gestión de SLURM}
\begin{infocaja}
\textbf{Componentes de SLURM}
\begin{itemize}
    \item \textbf{Controlador}
    \begin{itemize}
        \item Configuración de slurmctld
        \item Gestión de cuentas
        \item Políticas de colas
    \end{itemize}
    \item \textbf{Nodos de Cómputo}
    \begin{itemize}
        \item Instalación de slurmd
        \item Configuración de recursos
        \item Monitoreo de estado
    \end{itemize}
\end{itemize}
\end{infocaja}

\subsection{Gestión de Almacenamiento}
\begin{itemize}
    \item \textbf{NFS Server}
    \begin{itemize}
        \item Configuración de exportaciones
        \item Gestión de permisos
        \item Monitoreo de uso
    \end{itemize}
    \item \textbf{NFS Client}
    \begin{itemize}
        \item Montajes automáticos
        \item Gestión de caché
        \item Monitoreo de rendimiento
    \end{itemize}
\end{itemize}

\subsection{Autenticación}
\begin{infocaja}
\textbf{Implementación de OpenLDAP}
\begin{itemize}
    \item \textbf{Servidor LDAP}
    \begin{itemize}
        \item Instalación y configuración
        \item Certificados SSL
        \item Replicación
    \end{itemize}
    \item \textbf{Clientes}
    \begin{itemize}
        \item Configuración de autenticación
        \item Gestión de grupos
        \item Políticas de acceso
    \end{itemize}
\end{itemize}
\end{infocaja}

\subsection{Monitoreo}
\begin{itemize}
    \item \textbf{Prometheus}
    \begin{itemize}
        \item Configuración del servidor
        \item Reglas de alertas
        \item Retención de datos
    \end{itemize}
    \item \textbf{Grafana}
    \begin{itemize}
        \item Dashboards predefinidos
        \item Alertas visuales
        \item Exportación de informes
    \end{itemize}
\end{itemize}

% Capítulo 6: Mantenimiento del Proyecto
\chapter{Mantenimiento del Proyecto}
\section{Pipelines CI/CD}
\subsection{Automatización de Pruebas}
\begin{infocaja}
\textbf{Proceso de CI/CD}
\begin{itemize}
    \item \textbf{Integración Continua}
    \begin{itemize}
        \item Pruebas de roles Ansible
        \item Validación de sintaxis
        \item Pruebas de idempotencia
    \end{itemize}
    \item \textbf{Entrega Continua}
    \begin{itemize}
        \item Despliegue automatizado
        \item Verificación de cambios
        \item Rollback automático
    \end{itemize}
\end{itemize}
\end{infocaja}

\section{Infraestructura de Testing}
\subsection{Ambientes de Prueba}
\begin{itemize}
    \item \textbf{Desarrollo}
    \begin{itemize}
        \item Entorno virtualizado
        \item Pruebas de integración
        \item Desarrollo de nuevas características
    \end{itemize}
    \item \textbf{Staging}
    \begin{itemize}
        \item Réplica de producción
        \item Pruebas de rendimiento
        \item Validación de cambios
    \end{itemize}
\end{itemize}

\section{Documentación}
\subsection{Tipos de Documentación}
\begin{itemize}
    \item \textbf{Técnica}
    \begin{itemize}
        \item Arquitectura del sistema
        \item Procedimientos de operación
        \item Guías de troubleshooting
    \end{itemize}
    \item \textbf{Usuario}
    \begin{itemize}
        \item Guías de uso
        \item Tutoriales
        \item FAQ
    \end{itemize}
\end{itemize}

\section{Soporte y Mantenimiento}
\subsection{Procedimientos de Soporte}
\begin{infocaja}
\textbf{Niveles de Soporte}
\begin{itemize}
    \item \textbf{Nivel 1}
    \begin{itemize}
        \item Soporte básico de usuario
        \item Resolución de problemas comunes
        \item Documentación de incidencias
    \end{itemize}
    \item \textbf{Nivel 2}
    \begin{itemize}
        \item Soporte técnico avanzado
        \item Análisis de problemas complejos
        \item Coordinación con proveedores
    \end{itemize}
\end{itemize}
\end{infocaja}

% Apéndices
\begin{appendix}
\chapter{Glosario de Términos}
\section{Términos Técnicos}
\begin{itemize}
    \item \textbf{HPC}: High Performance Computing
    \item \textbf{SLURM}: Simple Linux Utility for Resource Management
    \item \textbf{NFS}: Network File System
    \item \textbf{LDAP}: Lightweight Directory Access Protocol
    \item \textbf{CI/CD}: Continuous Integration/Continuous Delivery
\end{itemize}

\chapter{Referencias Técnicas}
\section{Documentación Oficial}
\begin{itemize}
    \item Ansible Documentation: \url{https://docs.ansible.com}
    \item SLURM Documentation: \url{https://slurm.schedmd.com/documentation.html}
    \item OpenLDAP Documentation: \url{https://www.openldap.org/doc/}
    \item Prometheus Documentation: \url{https://prometheus.io/docs/}
\end{itemize}

\chapter{Documentación Adicional}
\section{Procedimientos Detallados}
\begin{itemize}
    \item Procedimientos de backup y recuperación
    \item Guías de actualización del sistema
    \item Procedimientos de seguridad
    \item Guías de optimización de rendimiento
\end{itemize}
\end{appendix}

\end{document} 